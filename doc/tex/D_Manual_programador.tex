\apendice{Documentación técnica de programación}

\section{Introducción}
En este anexo se presenta la documentación técnica de programación, que incluye una descripción de la estructura de directorios utilizada, una explicación para el futuro trabajo desarrollo de la aplicación

\section{Estructura de directorios}
La estructura de directorios utilizada que podemos ver en el repositorio es la siguiente:
\begin{itemize}
    \item \textbf{/}: Contiene ficheros de configuración, además del fichero README.
    \item \textbf{/CrowsFoot}: Directorio principal de la aplicación.
    \item \textbf{/CrowsFoot/editors}: Imágenes y ficheros de configuración, heredados de la librería.
    \item \textbf{/CrowsFoot/editors/config}: Ficheros de configuración en formato XML para trabajar con la librería mxGraph.
    \item \textbf{/CrowsFoot/editors/images}: Imágenes utilizadas en diferentes botones dentro de la aplicación.
    \item \textbf{/CrowsFoot/images}: Imágenes variadas para la aplicación.
    \item \textbf{/CrowsFoot/mxgraph}: Directorio principal de la librería mxGraph.
    \item \textbf{/CrowsFoot/src}: Directorio donde encontramos los elementos principales del proyecto.
\end{itemize}

\section{Manual del programador}
En el siguiente apartado se pretende explicar a futuros programadores como trabajar con la aplicación. Necesitaremos instalar los siguientes programas.

\subsection{IDE}
El entorno de trabajo que utilizaremos es un IDE que permita \emph{Javascript}, por lo que la mayoría de entornos de desarrollo nos pueden servir. En nuestro caso hemos utilizado \textbf{VSCode}.

\subsection{Git}
Necesitamos tener instalado \emph{Git} para hacer uso del control de versiones y poder obtener el repositorio de \emph{Github} de forma local realizando un clonado de este. \emph{Git GUI} nos permite trabajar con Git tanto con una interfaz de usuario común, como con línea de comandos como se acostumbra a trabajar con Git.

\subsection{Navegador}
Puesto que se trata de una aplicación web, necesitaremos un navegador moderno (Chrome, Firefox, Opera, etc.) que nos permita ejecutar la aplicación.

\section{Compilación, instalación y ejecución del proyecto}

\subsection{Compilación}
Javascript es un lenguaje el cual se compila en tiempo real o JIT\cite{wiki:jit}, por lo que no necesitamos ningún compilador, el navegador que utilicemos se encargará de realizar el compilado en tiempo real.

\subsection{Instalación}
Para instalar el proyecto clonaremos el repositorio desde \emph{Github}. Haremos uso de la herramienta Git GUI, tanto de su versión de interfaz como de línea de comandos, para obtener el repositorio. Además, se recomienda tener instalada la extensión \emph{Live Server}, o similares, la cual nos permite desplegar un servidor local donde se aloja nuestra aplicación. De esta forma podremos ver el comportamiento de la aplicación en un despliegue real, además de poder observar posibles errores que no sería posible de otra forma.

Con la interfaz gráfica simplemente completando los campos que nos pide la herramienta podremos obtener el repositorio.
\imagen{git_gui-clone}{Clonación con Git GUI}

Una vez importado el proyecto a nuestro sistema, podemos utilizar también esta herramienta para realizar las operaciones con Git que necesitemos, aunque no es indispensable, es otra opción que tiene esta herramienta.
\imagen{git_gui-cambios}{Git GUI}

Para clonar el repositorio mediante la línea de comandos utilizaremos el comando \emph{git clone}. Desde la línea de comandos accedemos al directorio donde deseemos colocar la copia del repositorio, introducimos el comando y la herramienta realizará la clonación.
\imagen{git_bash-clone}{Clonación Git Bash}

\subsection{Añadir características al proyecto}
Una vez obtenida una copia del repositorio, en caso de que queramos realizar cambios en la aplicación, crearemos una nueva rama desde la rama \emph{develop} con el nombre de la nueva funcionalidad a implementar y comenzaremos a trabajar desde ahí.

Una vez realizados los commits y pasando las pruebas de calidad necesarias, se incorporarán los cambios a la rama develop realizando un merge.

\subsection{Ejecución}

Para ejecutar el proyecto disponemos de dos formas.

\subsubsection{Preview Vercel}

Podemos utilizar las previews que nos proporciona \emph{Vercel}, las cuales nos muestran el última estado de la aplicación a partir del último push realizado al repositorio. La contra de utilizar este método es que necesitamos realizar el push al repositorio para poder visualizar los cambios, lo cual no nos permite ver los cambios que realizamos sobre la aplicación en tiempo real.

\imagen{preview-vercel}{Preview Vercel}

Para acceder la preview, podemos hacerlo desde el repositorio en \emph{Github}, en el apartado \emph{Deployments} donde nos llevará a la ventana que vemos en la parte superior, con un enlace al despliegue de la preview.

\subsubsection{Live Server}

\imagen{live-server}{Live Server}

\emph{Live Server} es una extensión del IDE VSCode que nos permite desplegar un servidor local con nuestra aplicación. Este servidor se recargará automáticamente cada vez que realicemos un cambio sobre el código, lo cual nos permite ver de forma instantánea el impacto de nuestros cambios sobre la aplicación.

Para desplegar el servidor, simplemente haremos click derecho sobre el fichero html y pulsaremos en la opción \textit{Open with Live Server}.

Como contra, este requerirá de algo más de potencia de nuestra máquina, pues es la encargada de ejecutar la aplicación.