\capitulo{7}{Conclusiones y Líneas de trabajo futuras}

En este apartado se comentan las conclusiones extraídas después de completar el proyecto, además de las posibles líneas de trabajo futuras por las que se podría continuar el proyecto más adelante.

\section{Conclusiones}

Las conclusiones obtenidas después del desarrollo de la aplicación son las siguientes:
\begin{itemize}
    \item La aplicación desarrollada cumple con el objetivo propuesto, permitiendo al usuario diseñar diagramas relacionales utilizando la notación de patas de cuervo (Crow's Foot), además de permitir al usuario importar y exportar sus diagramas en formato XML y obtener el código SQL y SQLAlchemy necesario para la creación de las tablas del diagrama.
    \item Se ha utilizado para todo el desarrollo la librería de Javascript \emph{mxGraph}. Esta librería utilizada para el diseño de grafos en aplicaciones web se lleva desarrollando desde el año 2005, por lo que se puede entender la profundidad de esta. Debido a esta profundidad, aunque se ha trabajado de forma intensiva con ella, considero que aún queda mucho por aprender y mejorar. Entendiendo el control que otorgaba la librería, a lo largo del desarrollo se decidió entregarle el control de la aplicación a mxGraph y no integrar ningún framework adicional. mxGraph ha permitido una implementación fluida y con buen rendimiento.
    \item Durante el desarrollo se han obtenido gran cantidad de nuevos conocimientos, principalmente en el desarrollo web y el lenguaje de programación Javascript, con el que no se había trabajado a este nivel con anterioridad. Este falta de trabajo previo probablemente haya afectado a la calidad del producto final, puesto que el aprendizaje ha sido continuo durante todo el desarrollo, mejorando paso a paso.
\end{itemize}

\section{Líneas de trabajo futuras}

Como posible línea de trabajo a seguir en versiones futuras de la aplicación, se podrían incluir las siguientes funcionalidades y mejores:
\begin{itemize}
    \item Implementar la gestión necesaria para las relaciones reflexivas, permitiendo al usuario dibujar estas.
    \item Introducir el manejo de múltiples relaciones entre las mismas dos tablas.
    \item Mejorar el manejo de claves de la aplicación, proporcionando al usuario más control sobre estas.
    \item Implementar un backend que permita al usuario registrarse y almacenar sus diagramas en la nube.
    \item Permitir la colaboración en tiempo real entre múltiples usuarios, siendo una herramienta colaborativa al estilo de \textit{Google Docs}.
    \item Ampliar las posibilidades de la generación de código, incluyendo soporte para mayor número de lenguajes de bases de datos, plataformas y librerías diferentes.
    \item Incluir una sección de \emph{Ayuda}, que guíe al usuario en sus primeros pasos en la aplicación, además de una sección de \emph{Soporte}, en caso de que algo salga mal.
    \item Introducir la IA en la aplicación, proporcionando al usuario una guía en tiempo real mientras desarrolla su diagrama, así como validación de que el diagrama es correcto.
    \item Ampliar la variedad de formatos en la que se pueden importar y exportar los diagramas, fomentando la compatibilidad con otras herramientas.
    \item Traducir la aplicación a diferentes idiomas, consiguiendo de esta forma que sea más accesible.
\end{itemize}