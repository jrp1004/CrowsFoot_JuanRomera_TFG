\capitulo{2}{Objetivos del proyecto}

A continuación, se detallan los diferentes objetivos que han motivado la realización del proyecto.

\section{Objetivos generales}
\begin{itemize}
    \item Desarrollar una aplicación web que permita diseñar bases de datos utilizando la notación de \emph{patas de cuervo}.
    \item Facilitar la creación de bases de datos generando código \emph{SQL} a partir del diagrama creado.
    \item Permitir exportar el diagrama creado en la aplicación.
\end{itemize}

\section{Objetivos técnicos}
\begin{itemize}
    \item Desarrollar una aplicación web utilizando la librería de Javascript \emph{mxGraph}.
    \item Añadir la capacidad de diseñar \emph{Diagramas E/R} desde la aplicación.
    \item Implementar la notación de \emph{Patas de Cuervo} para trabajar con los diagramas.
    \item Ser capaz de generar código SQL a partir de los diagramas creados en la aplicación.
    \item Desarrollar la posibilidad de exportar los elementos de mxGraph a un fichero XML.
    \item Comprobar que la aplicación funciona en una amplia gama de navegadores.
    \item Aplicar la metodología ágil Scrum en el desarrollo software.
    \item Utilizar Git como sistema de control de versiones, almacenando el repositorio en GitHub.
    \item Desplegar la aplicación para que pueda ser utilizada.
\end{itemize}

\section{Objetivos personales}
\begin{itemize}
    \item Aprender a manejarme en desarrollo web.
    \item Mejorar mis capacidades en el desarrollo de UI.
    \item Ampliar mis conocimientos en el diseño de \emph{Bases de Datos} y \emph{diagramas E/R}, incluyendo la, para mí, nueva notación de \emph{Patas de Cuervo}.
\end{itemize}