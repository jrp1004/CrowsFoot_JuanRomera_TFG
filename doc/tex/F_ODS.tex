\apendice{Anexo de sostenibilización curricular}

\section{Introducción}

La sostenibilidad es un pilar fundamental en el ámbito educativo y profesional. Si buscamos satisfacer nuestras necesidades a la vez que dejamos un mundo que puedan heredar nuestros hijos, necesitamos que una educación basada en el desarrollo sostenible, no sólo en el ámbito ecológico, sino también en el social y en el económico, se encuentre disponible para todos.

El desarrollo de este proyecto ha estado enfocado en la creación de una aplicación web que permita el diseño de diagramas relacionales utilizando la notación de patas de cuervo. Durante dicho desarrollo, se han tenido en cuenta el posible impacto que esta aplicación pudiera tener tanto en el ámbito ambienta, como en el social y económica, además de haber tenido siempre en mente su viabilidad a largo plazo, siguiendo los criterios generales para la sostenibilización curricular de la CRUE\cite{crue:sos-curr}.

Según dichos criterios, se han desarrollado los siguientes puntos.

\section{Sostenibilidad ambiental}

En cuanto a la \emph{sostenibilidad ambiental}, el desarrollo de una aplicación web presenta varios desafíos y oportunidades. Aunque los elementos con HTML y CSS no necesitan de muchos recursos, Javascript si que se trata de un lenguaje más demandante. Para el despliegue de la aplicación se ha utilizado la plataforma \textit{Vercel}, la cual cuenta con una política de energía verde con la que pretende reducir la huella de carbono producida por las aplicaciones que alberga\cite{ver:gr-pol}.

Durante el desarrollo se ha tratado de implementar el código más eficiente posible, tratando de reducir el número de instrucciones necesarias para ejecutar la aplicación con normalidad, reduciendo su posible impacto ambiental.

Al tratarse de una aplicación que se ejecuta desde el lado del cliente, no es necesario una comunicación constante con el servidor, lo que reduce en gran manera el consumo de recursos. Además, la posibilidad de importar y exportar los diagramas en ficheros XML favorece la reutilización de estos, evitando un uso excesivo de recursos en la aplicación realizando trabajo repetido.

\section{Sostenibilidad social}

Sobre la \emph{sostenibilidad social}, la accesibilidad de la aplicación es el punto más importante. La aplicación es de libre acceso para todo el mundo y desde todo el mundo, sin necesidad de pagos de ningún tipo. Además, el código de la aplicación se encuentra disponible para todo el mundo en su correspondiente repositorio, fomentando la prolongación del desarrollo.

\section{Sostenibilidad económica}

La \emph{sostenibilidad económica} en este proyecto guarda relación con el resto de puntos. Al hacer el código disponible para todo el mundo, se consigue más fácilmente un mantenimiento prolongado de la aplicación, por personas que estén dispuestas a ello. De esta forma, reducimos costes en mantenimiento y desarrollo y fomentamos que todo el mundo particípe en el desarrollo.

Además, habiendo desarralloda el código pensando en su eficiencia para una sostenibilida ambiental, consigue un menor uso de recursos, por lo que la aplicación resulta más fácil de mantener económicamente hablando.

\section{Inegración de la sostenibilidad en el currículum}

Es interesante de cara a tener un currículum más completo tener un trabajo previo en los criterios de sostenibilidad de la CRUE. De esta forma, podemos demostrar que somos capaces de, en la medida de lo posible, evaluar el impacto ambiental de nuestras aplicaciones, tener en cuenta la accesibilidad y usabilidad del usuario, conocer tecnologías sostenibles, desarrollar un proyecto de código abierto y estamos abiertos a continuar con la formación y sensibilización en este tema.

\section{Conclusiones}

El desarrollo de una aplicación web incluye tener en cuenta multitud de criterios de sostenibilidad, lo que implica una formación en la materia para poder aplicar dichos criterios de forma eficiente, mejorando la calidad del trabajo final además de ser más conscientes de la necesidad de un futuro más sostenible y equitativo.

En conclusión, la sostenibilidad implica tener un enfoque holístico que considere la eficiencia energética, la inclusión social y económica, incluyendo estos aspectos antes, durante y después del desarrollo para promover un desarrollo humano y ambientalmente sostenible.