\capitulo{6}{Trabajos relacionados}

\section{Artículos científicos}

\subsection{Basic Data Structure Models Explained With A Common Example}
Este artículo escrito en 1976, se trata de la primera aparición de las \emph{patas de cuervo}. Conocidas primero como \emph{flechas invertidas}, su autor, Gordon C. Everest, menciona que el artículo no trata sobre la notación, aunque esta fue escogida cuidadosamente. Conocida primero como \emph{fork} (tenedor), fue seleccionada por su facilidad para representarse mediante caracteres estándar (---<) y debido a que no implica ni direccionalidad ni un acceso físico.\cite{ge-bdsm}.

\section{Proyectos}

\subsection{mxGraph Schema}
Ejemplo de los creadores de la librería \emph{mxGraph}, muestra las funcionalidades báscias para implementar una aplicación de diseño de diagramas de bases de datos. La aplicación desarrollada toma como base este ejemplo, mejorándolo y ampliándolo con la notación de patas de cuervo. En el ejemplo únciamente disponemos de relaciones direccionales sin cardinalidad y un simple manejo de claves foráneas añadiendo la clave primaria de la tabla con la que se relaciona. En la aplicación desarrollada se han extendido las funcionalidades, tanto en las relaciones con la nueva notación, como en las columnas ampliando sus atributos.

\begin{itemize}
    \item Web del ejemplo: \url{https://jgraph.github.io/mxgraph/javascript/examples/schema.html}
\end{itemize}

\subsection{Microsoft Visio}
La aplicación \emph{Visio} de Microsoft es utilizada para el diseño de diagramas y gráficos vectoriales. Esta aplicación nos permite, entre otros, diseñar diagramas entidad-relación disponiendo de diferentes notaciones. Entre las notaciones que podemos utilizar encontramos la notación de patas de cuervo, aquí denominada notación de patas de gallo. Con Visio podemos diseñar nuestros diagramas, añadiendo las tablas, columnas y relaciones que necesitemos. Microsoft Visio cuenta con diferentes licencias, gratuitas y de pago, lamentablemente, la notación de patas de cuervo se encuentra únicamente disponible en la versión de pago.

\begin{itemize}
    \item Web de la aplicación: \url{https://www.microsoft.com/es-es/microsoft-365/visio/flowchart-software}.
\end{itemize}

\subsection{draw.io}
\emph{draw.io} es una aplicación web pensada para el diseño de multitud de diagramas. Esta aplicación, desarrollada por \emph{JGraph} los creadores de la librería \emph{mxGraph}, también nos permite diseñar diagramas entidad-relación utilizando la notación de patas de cuervo, teniendo a disposición las tablas y columnas que representarán a las entidades y sus atributos, y todas las posibles relaciones que podemos encontrar entre las tablas.

\begin{itemize}
    \item Web de la aplicación: \url{https://app.diagrams.net/}
\end{itemize}

\subsection{Visual Paradigm}
\emph{Visual Paradigm} cuenta con una herramienta online para diseñar infinidad de gráficos y diagramas. Entre la multitud de opciones de las que dispone, podemos diseñar diagramas E/R utilizando la notación de patas de cuervo. Aunque las opciones que dispone son algo limitadas, es una buena opción a considerar teniendo en cuenta la posibilidad de trabajar online y su acceso gratuito.

\begin{itemize}
    \item Web de la aplicación: \url{https://online.visual-paradigm.com/es/}
\end{itemize}

\section{Fortalezas y debilidades del proyecto}
