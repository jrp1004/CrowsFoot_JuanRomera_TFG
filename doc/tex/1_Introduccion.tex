\capitulo{1}{Introducción}

El uso de los sistemas de bases de datos está ampliamente extendido en diferentes ámbitos con múltiples usos. Dentro de los diferentes tipos de bases de datos nos encontramos las bases de datos relacionales\cite{wiki:relational_db}. Un modelo comúnmente utilizado para describir este tipo de bases de datos es el modelo relacional (relational model)\cite{wiki:relational_model}, modelo compuesto por relaciones, las cuales representan la información con la que trabaja la base de datos, y sus atributos.

Entre las notaciones comúnmente utilizadas para representar las relaciones entre las entidades de la base de datos encontramos la notación Crows Foot o patas de cuervo, objeto de este trabajo. Esta notación, que recibe este nombre debido a la similitud de los extremos de las líneas que representan las relaciones con las patas de un cuervo, utiliza tres símbolos para representar la cardinalidad de las relaciones que se utilizan en pares para representar los cuatro tipos de cardinalidad que una entidad podría tener en una relación.

\section{Estructura de la memoria}
La memoria se estructura de la siguiente forma:
\begin{itemize}
    \item \textbf{Introducción: } descripción del problema a resolver y la solución propuesta. Estructura de la memoria y listado de materiales adjuntos.
    \item \textbf{Objetivos del proyecto: } exposición de los objetivos que se persiguen con el proyecto.
    \item \textbf{Conceptos teóricos: }breve explicación de los conceptos teóricos clave para la comprensión de la solución propuesta.
    \item \textbf{Técnicas y herramientas: }listado de técnicas metodológicas y herramientas utilizadas para gestión y desarrollo del proyecto.
    \item \textbf{Aspectos relevantes del desarrollo: }exposición de aspectos destacables que tuvieron lugar durante la realización del proyecto.
    \item \textbf{Trabajos relacionados: }estado del arte en el campo del diseño de aplicaciones web para la realización de diagramas de bases de datos.
    \item \textbf{Conclusiones y líneas de trabajo futuras: }conclusiones tras la realización del proyecto y posibilidades de mejora o expansicón de la solución aportada.
\end{itemize}

\section{Estructura de los anexos}
Junto a la memoria se proporcionan los siguientes anexos:
\begin{itemize}
    \item \textbf{Plan del proyecto software: }planificación temporal y estudio de viabilidad del proyecto.
    \item \textbf{Especificación de requisitos del software: }se describe la fase de análisis; los objetivos generales, el catálogo de requisitos del sistema y la especificación de requisitos funcionales y no funcionales.
    \item \textbf{Especificación de diseño: }se describe la fase de diseño; el ámbito del software, el diseño de datos, el diseño procedimental y el diseño arquitectónico.
    \item \textbf{Manual del programador: }recoge los aspectos más relevantes relacionados con el código fuente (estructura, compilación, instalación, ejecución, pruebas, etc.).
    \item \textbf{Manual de usuario: }guía de usuario para el correcto manejo de la aplicación.
\end{itemize}

\section{Materiales adjuntos}

Los materiales que se adjuntan con la memoria son:

\begin{itemize}
    \item Memoria
    \item Anexos
    \item Código fuente y recursos de la aplicación
    \item Vídeo presentación y vídeo tutorial
    \item Enlaces a repositorio y a aplicación de proceso de calidad
\end{itemize}